
\chapter{Mathematics}
% Alternative title: kill me now. 
% I will buy the person that writes this section a Freddo. - Matt Bessey.
% Matt, you owe me a Freddo - Drum Ogilvie
    \section{Modular Arithmetic \& Groups}
    \subsection{Modular Arithmetic}
    Modular arithmetic is a bit odd at first, but you get used to it. It's all about remainders, really; we express modular relations in the following form:\\
    \begin{figure}[htp!]
    \centering
        $$ X \equiv \textit{Y \emph{\textbf{MOD}} Z} $$
    \end{figure}\\
    Where $X$ is any old number and $Z$ is the `Modulus'. Y is the remainder of the integer division of $X$ by $Z$, the result of an operation expressed in many languages (including C) as $X \% Z$. Simples.\\
    \\
    When you are doing modulo maths in crypto, you only really need to remember 1 rule:
    $$(a+b) \mod N \equiv \big[(a \mod N) + (b \mod N)\big] \mod N$$
    I've used addition when writing that rule down, but it's the same for subtraction and multiplication. The reason for this is that addition and multiplication \textbf{group operations} on the set of modulo remainders. SAY WHAT?\\
    \\
    This leads us nicely onto groups.

    \subsection{Groups}
    A mathematical group is a combination of a set and an operation, usually written $(G,*)$. We require three properties of this operation;
	\begin{align*}
		\mbox{Closure: } &\forall x,y \in G \quad x*y \in G\\
		\mbox{Associativity: } &\forall x,y,z \in G \quad (x*y)*z = x*(y*z)\\
		\mbox{Identity: } &\exists e \in G \; st \; \forall x \in G \quad e*x = x*e = x \\
		\mbox{Invertability: } &\forall x \in G \; \exists x^{-1} \in G \;\; st \;\; x*x^{-1} = x^{-1} * x = e
	\end{align*}
	Most cryptographic constructions used here are defined on the group of non-zero integers modulo $N$ under modular multiplication, defined thusly:
    \begin{align*}
        \mathbb{Z}/N\mathbb{Z}^{*} &= \{1, 2, 3, ... , N-1\} \\
        \forall x,y, \in \mathbb{Z}/N\mathbb{Z}^{*}& \quad x * y = xy \mod N
    \end{align*}
    This is actually only a group if $N$ is prime. You should think about why $0$ is not a member of this group, and then why, if $N$ is not prime, this is not a well-defined group. We'll deal with this later when we talk about rings.\\
\\
	Whilst the only concrete we'll deal with here is $(\mathbb{Z}/N\mathbb{Z} , \cdot)$, other groups exist and are used in the real world, and cryptographic constructions and problems are often defined over a generic group, rather than over this specific one.\\
\\
	If we have the extra special benefit of commutativity in our operation, ie $x*y = y*x$, then this is an `\textbf{Abelian Group}' (sometimes just called a `commutative group').\\
\\
    Groups can sometimes be defined in terms of \textbf{generators}. We say that $g \in G$ is a generator for $(G,*)$ iff we can rebuild $G$ exclusively from $g^n$, with various $n$, where $n$ is a natural number. This differs from the idea of a generating set which does not appear to be part of this course, but if you're interested, then we say that $S\subseteq G$ is a generating subset of $G$ iff $G$ can be expressed using only finite products of elements of $S$ and their inverses.

    \subsection{Rings}
    In fact, we can define both the addition and multiplication operations in $\mathbb{Z}/N\mathbb{Z}$. Rather than defining two seperate groups, we combine the two into a new thing called a `ring'. A ring in this context is a set with operations for addition and multiplication, with all the properties you might expect.\\
    \\
    It's like an abelian group with respect to addition, because addition is closed, associative, commutative, has an identity and a guaranteed inverse. Not only that, my mentally-endowed comrades, but it is like an abelian group for multiplication too, with the added awesome that multiplication is distributive over addition.\footnote{The minimal spec for a ring is an abelian group with a second associative binary operation that is distributive over the main group operation.}

    \subsection{Multiplicative Inverse}
    The `inverse' for a group member is another member of the same group that when used as the second argument for the operation gives the result of the identity for that operation.\\
    \\
    In this case, the operation is multiplication, the multiplicative identity is 1, and our group is usually numbers Modulo N. So the inverse of one number Mod N is another number less than N, that when you multiply them together you get a number which is 1 Mod N. Still here? Still awake? Stay with me, soldier.
    
    \section{Greatest Common Divisor}
    Say you've got two numbers. What is the biggest factor that they both share? BOOM. DONE.\\
    \\
    Ok, well, there's probably this really handy algorithm you should know; the \textbf{Euclidean algorithm}. Invented by \sout{Nigel Smart} Euclid bloomin' ages ago, it's a fairly simple way to efficiently figure out the GCD of 2 numbers.
        \begin{itemize}
            \item Let $k$ be a number, starting at 0, that is incremented at each stage of this algorithm
            \item Let $r_{-2} = a, r_{-1} = b$, where $a$ and $b$ are the two numbers you are testing, and $a$ is the larger of the two.
            \item A stage is this: 
                $$
                    r_{k-2} = q_{k} \cdot r_{k-1} + r_{k}
                $$
            \item Where $q_{k}$ is the quotient of the integer division $\frac{r_{k-2}}{r_{k-1}}$, and $r_{k}$ is the remainder.
            \item Keep going until $r_{k} = 0$, at which point $r_{k-1}$ is the GCD. If $r_{k-1}$ is 1, then your original numbers are \textbf{co-prime} (aka \textbf{relatively prime} or \textbf{coprime}), meaning that they have no common factors.
        \end{itemize}
    
    \section{Euler Phi Function}
    Also known as Euler's Totient Function. Because why the hell not. It is simply a measure of how many members of our set of remainders are \textbf{relatively prime} to N. In other words, how many members share no common factors with N. If we choose N to be prime, which we normally do, then $\phi(N) = N - 1$. Easy, right?\\
    \\
    The other important case for cryptography is where we are using two primes ($p$ and $q$) to create a new modulus:
    $$ \phi(p \cdot q) = (p-1)(q-1)$$
    Since you are undoubtedly on a roll, here is the actual formula (it's not too scary); the first is the prime factorization of N, which we need.

    $$N = \prod_{i=1}^{n}p_{i}^{e_{i}} \textbf{ allowing us to use } \phi(N) = \prod_{i=1}^{n}p_{i}^{e_{i}-1}(p_{i} - 1)$$


    \section{Lagrange's Theorem}
        % TODO: Not sure if this is even correct. It certainly doesn't cover the most important bit of Lagrange' theorem concering subgroups. - Matt
        Lagrange's theorem states for a group $G$ of order (size) $n$ then for all $a \in G$:
        $$a^{n} = 1$$
        So in a group modulus under addition, the following also holds: 
        $$ a^{\phi(N)} \equiv 1 \mod N$$ 
    
        \subsection{Fermat's Little Theorem}
            Fermat's little theorem is a special case of Lagrange's theorem, and a little theorem it is. It simply states that, if $p$ is a prime number, then for all $a$:

            $$ a^p \equiv a \mod p $$
        
    \section{Fields}
        A field is a special variety of \textbf{abelian ring}. Whereas a ring's second (multiplicative) operation doesn't need to have an inverse for every non-zero element, a field's does. 
    
    \section{Chinese Remainder Theorem}
        The Chinese Remainder Theorem, often written CRT, states that given a modulus $N$, made up of pairwise coprime factors $n_1$, ..., $n_r$, it is possible to calculate $x = y \mod N$ by instead calculating modulo its component parts, $n_1$, ..., $n_r$ and the combining those parts using some clever maths we'll get onto in a moment. This isn't particularly useful with small $N$, but you can imagine when your $N$ and $x$ are in the $2^{1024}$ range, and you're performing modular exponentiation it has some performance implications.\\
        \\
        Anyway, back to the maths... The CRT states that $x = y \mod N$ can be reconstructed based on answers to each sub-modulus thus:

        $$ x = \sum\limits_{i=1}^{r} a_i N_i y_i \mod N $$
        Where $a_i$ is the remainder modulo $n_i$, $N_i = N/n_i$, and $y_i = N_i^{-1} \mod n_i$\\
        \\
        The significant thing to note here is that this means if we know what $x$ is equal to modulo the component parts of $N$, then we know what $x$ is equal to modulo $N$.\footnote{And presumably vice versa, though not so sure about this.}

    
    \section{One Way Functions}
        Firstly, lets define a function simply as something that, for items in set $X$, provides a mapping to items in set $Y$:
        $$f\colon X \rightarrow Y : x \mapsto y = f(x)$$
        Such a function is known as a \textbf{one way function} if, for all $x$, $f(x) = y$ is efficient to compute, while computing the $x$ that maps to $y$ is infeasible. When we say infeasible, we mean its so bloody hard it would essentially take, like, \textbf{forever}.\\
        \\
        There is a special subset of one way functions, known as \textbf{trapdoor one way functions}. These are so named because, while reversing them naively is infeasible, given some extra \textbf{trapdoor information}, they are \textbf{easily inverted}. An example of such a function would be the RSA algorithm, where deducing $\cm$ from $\cc$ is impossible without the necessary trapdoor information, which in the case of RSA is the private key.

    \section{SQROOT Problem}
        The square root problem, often written SQROOT, is that it is bloody difficulty to find $x$ where:
            $$y \mod N =  x * x \mod N $$
        Even when $y$ and $N$ are known, without knowing $N$'s factors.

    
    %%%%%%%%%%%%%%%%%%%%%%%%%%%%%%%%%%%%%%%%
    % There's actually a few more sections %
    %%%%%%%%%%%%%%%%%%%%%%%%%%%%%%%%%%%%%%%%