\chapter{Problems Class}

	\section{Problems Class 1}
		$\cdots$

	\section{Problems Class 2}
	\begin{enumerate}
		\item Flip it!\\\\
			\begin{tabular}{lllllllll}
				\toprule
				$x$&1&2&3&4&5&6&7&8\\
				\midrule
				$\pi(x)$&2&4&6&1&8&3&5&7\\
				\bottomrule
			\end{tabular}

			\begin{verbatim}
			GENTLEMANDONODRE
			\end{verbatim}

		\item N chained one-to-one substitutions can always be replaced with a functionally identical single substitution. Therefore no more secure.

		\item Same as above

		\item Sub and perm commutative so $S(P(S(P(m)))) = S(S(P(P(m))) \equiv S(P(m))$.
\comment{
		\item 
		\begin{itemize}
			\item XORs are commutative by themselves, when there are perms it is harder: you must permute the key. XOR and subs are not commutative.\\
			\\
			\item 
		\end{itemize}
}
	\end{enumerate}

\section{Problem Class 3}
	\begin{itemize}
		\item IND-CPA must be OW-CPA. The problem of OW-CPA is to decrypt a ciphertext with an encryption oracle. The problem of IND-CPA is to distinguish one ciphertexts from two messages given an encryption oracle. If we have a method to break OW-CPA then we can use this to break IND-CPA, therefore IND-CPA can be reduced to OW-CPA.
		\item No
		\item :S
		\item Add stuff on end, decrypt, get rid of stuff. Because the encryption only uses the context of the previous 
		\item 
	\end{itemize}